%-------------------------
% Resume in Latex
% Original Author : Sourabh Bajaj
% Adaptation : Hyunggi Chang
% License : MIT
%------------------------

\documentclass[letterpaper,11pt]{article}

\usepackage{latexsym}
\usepackage{kotex} % 한글 사용 가능! 
\usepackage[empty]{fullpage}
\usepackage{titlesec}
\usepackage{marvosym}
\usepackage[usenames,dvipsnames]{color}
\usepackage{verbatim}
\usepackage{enumitem}
\usepackage[hidelinks]{hyperref}
\usepackage{fancyhdr}
\usepackage[english]{babel}
\usepackage{tabularx}
\usepackage{amsmath}
\usepackage{kotex} % Enable Korean!

\pagestyle{fancy}
\fancyhf{} % clear all header and footer fields
\fancyfoot{}
\renewcommand{\headrulewidth}{0pt}
\renewcommand{\footrulewidth}{0pt}

% Adjust margins
\addtolength{\oddsidemargin}{-0.5in}
\addtolength{\evensidemargin}{-0.5in}
\addtolength{\textwidth}{1in}
\addtolength{\topmargin}{-0.5in}
\addtolength{\textheight}{1.0in}

\urlstyle{same}

\raggedbottom
\raggedright
\setlength{\tabcolsep}{0in}

% Sections formatting
\titleformat{\section}{
  \vspace{-4pt}\scshape\raggedright\large
}{}{0em}{}[\color{black}\titlerule \vspace{-2pt}]

%-------------------------
% Custom commands
\newcommand{\resumeItem}[1]{
  \item\small{
    {#1 \vspace{-2pt}}
  }
}

\newcommand{\resumeSummary}[1]{
  \item
    \begin{tabular*}{0.97\textwidth}[t]{l@{\extracolsep{\fill}}r}
      #1
    \end{tabular*}
}

\newcommand{\resumeSubheading}[4]{
  \vspace{-1pt}\item
    \begin{tabular*}{0.97\textwidth}[t]{l@{\extracolsep{\fill}}r}
      \textbf{#1} & #2 \\
      \textnormal{\small #3} & \textnormal{\small #4} \\
    \end{tabular*}\vspace{-5pt}
}

\newcommand{\resumeEmployment}[4]{
  \vspace{-1pt}\item
    \begin{tabular*}{0.97\textwidth}[t]{l@{\extracolsep{\fill}}r}
      \textbf{#1} & #2 \\
      \textnormal{\small #3} & \textnormal{\small #4} \\
    \end{tabular*}\vspace{-5pt}
}

\newcommand{\resumeProject}[2]{
  \vspace{-1pt}\item
    \begin{tabularx}{0.97\textwidth}[t]{X}
      	\textbf{#1} \\
      {\small\raggedright #2} \\
    \end{tabularx}\vspace{-5pt}
}

\newcommand{\resumeResearch}[5]{
  \vspace{-1pt}\item
    \begin{tabular*}{0.97\textwidth}[t]{l@{\extracolsep{\fill}}r}
      \textbf{#1} & #2 \\
      \textit{\small#3} {\small #4 \vspace{-2pt}} & \textit{\small #5} \\
    \end{tabular*}\vspace{-5pt}
}

\newcommand{\resumeTalk}[2]{
  \vspace{-1pt}\item
    \begin{tabular*}{0.97\textwidth}[t]{l@{\extracolsep{\fill}}r}
      \textbf{#1} & #2 \\
    \end{tabular*}\vspace{-5pt}
}

\newcommand{\resumeSkills}[1]{
  \item
    \begin{tabular*}{0.97\textwidth}[t]{l@{\extracolsep{\fill}}r}
      #1
    \end{tabular*}
}

\newcommand{\resumeCommunity}[3]{
  \vspace{-1pt}\item
    \begin{tabular*}{0.97\textwidth}[t]{l@{\extracolsep{\fill}}r}
      \textbf{#1} & #2 \\
      \textit{\small#3} \\
    \end{tabular*}\vspace{-5pt}
}

\newcommand{\resumeSubItem}[2]{\resumeItem{#1}{#2}\vspace{-4pt}}

\renewcommand{\labelitemii}{$\circ$}

\newcommand{\resumeSubHeadingListStart}{\begin{itemize}[leftmargin=*]}
\newcommand{\resumeSubHeadingListEnd}{\end{itemize}}

\newcommand{\resumeEmploymentListStart}{\begin{itemize}[leftmargin=*]}
\newcommand{\resumeEmploymentListEnd}{\end{itemize}}

\newcommand{\resumeItemListStart}{\begin{itemize}}
\newcommand{\resumeItemListEnd}{\end{itemize}\vspace{-5pt}}

%-------------------------------------------
%%%%%%  CV STARTS HERE  %%%%%%%%%%%%%%%%%%%%%%%%%%%%


\begin{document}

%----------HEADING-----------------
\begin{tabular*}{\textwidth}{l@{\extracolsep{\fill}}r}
  \textbf{\href{https://lshtar13.github.io}{\Large 이해성}} & Github: \href{https://github.com/lshtar13}{lshtar13}  \\
  \href{https://lshtar13.github.io}{https://lshtar13.github.io} & Email : \href{mailto:edwin109802@gmail.com}{edwin109802@gmail.com} \\
  {} & Mobile : (+82) 10-5641-3844
\end{tabular*}

%-----------Summary-----------------
\section{Summary}
  \resumeSubHeadingListStart
    \resumeSummary{원리에 대한 이해가 가장 유용한 디버깅이라고 생각하는 개발자입니다. 백엔드 엔지니어링과 시스템 프로그래밍에 많은 관심을 가지고 공부해 나가고 있습니다.}
    \item \textbf{주요 역량 요약}
    \begin{itemize}
      \resumeItem{NestJS를 이용한 백엔드 API 서버 개발.}
      \resumeItem{Go, Python, Node.js를 이용한 비동기 동시성 프로그래밍.}
      \resumeItem{RabbitMQ 등 오픈소스를 이용한 메시지 큐 및 캐시 시스템 설계 및 구현.}
      \resumeItem{GDB 등을  이용한 시스템 소프트웨어 및 오픈소스 분석 및 디버깅.}
    \end{itemize}
    \item \textbf{주요 경험 요약}
    \begin{itemize}
      \resumeItem{2번의 인턴십, 1번의 학부연구생, 3번의 협업 프로젝트 개발 경험.}
      \resumeItem{TypeScript를 사용해 NestJS 기반 실사용 서비스의 REST 및 GraphQL API 개발 경험.}
      \resumeItem{Go와 RabbitMQ, Redis를 사용해 대용량 트래픽을 처리하는 마이크로서비스 유지보수 및 기능 개발 경험.}
      \resumeItem{Python과 FastAPI, gRPC, Kafka를 사용해 다수의 트레이딩 봇을 운영하는 플랫폼 개발 경험.}
      \resumeItem{Bruno 등을 이용한 API 문서화 및 Prisma, SQLAlchemy 등 ORM을 이용한 데이터베이스 설계 및 스키마 작성 경험.}
    \end{itemize}
  \resumeSubHeadingListEnd


%-----------EXPERIENCE-----------------
\section{Experience}
  \resumeEmploymentListStart
    \resumeEmployment
      {신한투자증권 블록체인부 인턴}{2025.10 - 현재}
      {암호화폐 트레이딩 봇 플랫폼 MVP 개발}{}
        \resumeItemListStart
            \resumeItem{암호화폐 거래소의 API를 활용해 고빈도 매매를 수행하는 트레이딩 봇들을 구동하고 관리하는 플랫폼을 기획하고 구현.}
            \resumeItem{Python의 asyncio로 트레이딩 봇을 구동하는 비동기 gRPC 서버 구축.}
            \resumeItem{kafka를 도입해 주문 직렬화 및 확장성을 고려한 주문 처리 마이크로서비스를 개발하였고 gRPC 서버와의 이벤트 기반 아키텍쳐를 구성함.}
            \resumeItem{암호화폐 및 DeFi 생태계를 조사하여 수익을 낼 수 있는 차익거래 전략을 수립 및 트레이딩 봇으로 구현.}
        \resumeItemListEnd   
      {예치금 관리 서비스 QA 대응}{}
        \resumeItemListStart
            \resumeItem{조각투자 서비스의 제휴사들에 대한 계좌 개설 등의 QA 요청 처리.}
        \resumeItemListEnd   
    
        
    \resumeEmployment
      {성균관대학교 지능형임베디드시스템연구실 학부연구생}{2023.08 - 2024.02}
      {\href{https://www.usenix.org/conference/atc17/technical-sessions/presentation/park}{iJournaling}기반 ext4 파일시스템의 fsync 시스템 콜 성능 개선 연구}{}
      \resumeItemListStart
          \resumeItem{리눅스 ext4 파일시스템 상 fast-commit으로 구현되어 있는 \href{https://www.usenix.org/conference/atc17/technical-sessions/presentation/park}{iJournaling} 기법을 fsync() 호출 패턴에 따라 선택적으로 적용하는 기법을 고안하고 커널 상에 구현함.}
          \resumeItem{Ext4를 gdb를 이용해 동적으로 분석하여 write() 및 fsync()의 로직을 분석함.}
          \resumeItem{Femu를 이용한 다양한 file I/O workload 시험 및 다수의 파일에 대한 write()이 이루어지는 환경에서의 iJournaling의 취약성 파악.}
          \resumeItem{fsync() 호출 시 I/O 크기와 이전 호출과의 간격을 고려하여 선택적으로 iJournaling을 수행하는 로직을 구현.}
          \resumeItem{8KiB에서 4MiB에 이르는 다양한 크기와 패턴의 workload 벤치마크를 활용하여 선택적 iJournaling 로직을 테스트하였을 때, 대부분의 경우에서 세 개의 로직 중 가장 빠르거나 두번째로 빠른 것을 확인함.}
      \resumeItemListEnd
    
    \resumeEmployment
      {(주)소이넷 산학 인턴}{2023.06 - 2023.08}
      {오픈소스 CV 모델 포팅}{}
      \resumeItemListStart
          \resumeItem{YoLov7, YoLov8 등 오픈소스 CV 모델을 CUDA 기반 당사 솔루션으로 포팅하는 작업 수행.}
          \resumeItem{Python, PyTorch로 작성된 모델의 구조를 분석하고, C++/CUDA 기반 단일 추론 함수로 단순화함.}
          \resumeItem{한국인 안면 데이터를 전처리해 안면 인식 모델인 MagFace에 학습시키는 작업 수행.}
      \resumeItemListEnd
      
  \resumeEmploymentListEnd

%-----------Projects-----------------

\section{Projects}
  \resumeSubHeadingListStart
    \resumeProject
      {Codedang (\href{https://github.com/skkuding/codedang}{github.com/skkuding/codedang})}
         { 성균관대 자체 코딩 플랫폼 구축을 목표로 하는 프로젝트입니다. 
         교내 학생동아리리 SKKUDING에서 개발 및 유지보수를 담당하고 있으며, 이러한 공로를 인정받아 2023년 성균명품스터디클럽에 선정되었습니다. 
         현재는 온라인 저지 서비스를 지원하며 각종 프로그래밍 수업과 대회에 활용되고 있습니다.}
         \resumeItemListStart
          \resumeItem{NestJS로 작성된 백엔드 서비스의 기능 개발과 유지보수를 담당함.}
          \resumeItem{REST API로 client 서비스와 통합되어 제공되던 admin 서비스를 GraphQL로 전환 및 분리함.}
          \resumeItem{Prisma ORM을 이용해 데이터베이스 스키마를 설계하고, Bruno로 API 문서화 및 테스트를 수행함.}
          \resumeItem{Mocha와 Chai를 사용해 단위 및 통합 테스트를 작성함.}
          \resumeItem{Go로 작성된 judger 서비스와 Redis, RabbitMQ로 구성된 마이크로서비스의 기능 개발 및 유지보수를 담당함.}
          \resumeItem{Go 고루틴을 활용해 순차 채점 로직을 병렬 처리 구조로 변경하여 동시 채점 처리량(Throughput) 증대, NestJS ↔ RabbitMQ 통신을 비동기적으로 개선함.}
          \resumeItem{Judger에 스페셜 저지 기능을 추가함.}
         \resumeItemListEnd
    
    \resumeProject
      {Fin2Vec (\href{https://github.com/HOYNET/Fin2Vec}{github.com/HOYNET/Fin2Vec})}
      { 주식 시세 및 재무제표 데이터를 딥러닝 모델을 사용해 임베딩으로 변환하여 각 주식 종목별 연관도를 분석하는 모델을 개발한 프로젝트입니다.
      파악된 연관도를 바탕으로 포트폴리오 추천 및 주가 예측 시스템을 개발하였습니다.
        해당 프로젝트는 성균관대학교 대학혁신과공유센터의 2023학년도 2학기 Co-Deeplearning 프로젝트에 선정되어 지원받아 진행하였습니다.}
      \resumeItemListStart
        \resumeItem{트랜스포머 모델의 인코더를 도입해 주식 시세 및 재무제표 데이터를 임베딩으로 변환하는 모델을 설계함.}
        \resumeItem{PyTorch를 활용해 모델을 학습하고, 적절한 임베딩을 생성하는 데이터셋을 선정함.}
        \resumeItem{추출된 임베딩을 블룸버그의 종목간 연관도 데이터로 검증하고 클러스터링 기법을 고안함.}
        \resumeItem{연관도 정보를 바탕으로 포트폴리오 추천 시스템을 설계 및 구현함.}
      \resumeItemListEnd
    
    \resumeProject
      {Gradphone (\href{https://github.com/jspark2000/skku-qr-backend}{github.com/jspark2000/skku-qr-backend})}
    {졸업식 및 각종 학교 행사 시 QR코드와 TTS를 통해 프레젠테이션을 지원하는 시스템을 개발한 프로젝트입니다. }
      \resumeItemListStart
        \resumeItem{NestJS와 Prisma ORM을 사용해 백엔드 API 서버를 개발.}
        \resumeItem{구글 TTS와의 연동을 진행함.}
      \resumeItemListEnd

  \resumeSubHeadingListEnd

%-----------Skills-----------------

\section{Skills}
  \resumeSubHeadingListStart
    \resumeSkills{\textbf{Programming Languages} - Typescript, Python, Go, C}
    \resumeSkills{\textbf{Frameworks \& Libraries} - NestJS, gRPC, GraphQL, PyTorch}
    \resumeSkills{\textbf{Others} - Docker, Prisma, SQLAlchemy, GDB}
  \resumeSubHeadingListEnd

%-----------EDUCATION-----------------
\section{Education \& Others}
  \resumeSubHeadingListStart
    \resumeSubheading
      {성균관대학교}{수원, 경기}
      {소프트웨어학과 -- 학점(전공): 4.14(4.34)/4.5}{2022.02 -- 현재}
    \resumeSubheading
      {대한민국 육군}{}
      {병장 만기 전역}{2024.04 -- 2025.10}
  \resumeSubHeadingListEnd

% %-----------EXPERIENCE-----------------
% \section{Research Experiences}
%   \resumeSubHeadingListStart

%     \resumeResearch
%       {[Lab Name], [School name]}{Seoul, South Korea}
%       {[Position] (Advisor: [Name of Lab head])}{}{Jan. 20XX - Mar. 20XX}
%     \resumeResearch
%       {[Lab Name], [School name]}{Seoul, South Korea}
%       {[Position] (Advisor: [Name of Lab head])}{}{Jan. 20XX - Sept. 20XX}

%   \resumeSubHeadingListEnd

% %-----------Talks-----------------
% \section{Conference Talks}
%     \resumeSubHeadingListStart
    
%         \resumeTalk{[Conference name]}{20XX}
%             \resumeItemListStart
%                 \resumeItem{[Talk details]}
%             \resumeItemListEnd

%         \resumeTalk{[Conference name]}{20XX}
%             \resumeItemListStart
%                 \resumeItem{[Talk details]}
%             \resumeItemListEnd

%     \resumeSubHeadingListEnd
    
% %-----------Papers-----------------
% \section{Papers}
%     \resumeSubHeadingListStart
%         \resumeItem{[\textbf{Paper Name}] [Authors] [Journal Name] [Other citation details}
%         \resumeItem{[\textbf{Paper Name}] [Authors] [Journal Name] [Other citation details}
%         \resumeItem{[\textbf{Paper Name}] [Authors] [Journal Name] [Other citation details}
%     \resumeSubHeadingListEnd

% %-----------Community-----------------
% \section{Community}
%     \resumeSubHeadingListStart
%         \resumeCommunity{[Role]}{20XX -- Present}
%         {[Community Name]}
        
%         \resumeCommunity{[Role]}{20XX -- Present}
%         {[Community Name]}
%         \resumeItemListStart
%             \resumeItem{[Some Event information]}
%             \resumeItem{[Some Group information}
%         \resumeItemListEnd    

%     \resumeSubHeadingListEnd

%-------------------------------------------
\end{document}
